\documentclass[a4paper]{article}

%\documentclass[smallextended]{svjour3} 
\usepackage{amsmath}
\usepackage{amssymb}
\usepackage{stmaryrd}
\usepackage{url}
\usepackage{pgf, tikz}
\usepackage{mathrsfs}
\usetikzlibrary{arrows, arrows.meta, decorations.pathmorphing, backgrounds, fit, positioning, shapes.symbols, shapes.geometric, chains}
\usepackage[T1]{fontenc}
\usepackage{csquotes}
\usepackage{scalerel}
\usepackage{float, subfig}
\usepackage{booktabs}

\usepackage{comment}

\newcommand{\lbrakk}{\llbracket}
\newcommand{\rbrakk}{\rrbracket}
\newcommand{\tab}{\hspace{5mm}}
\DeclareMathOperator{\arccosh}{arccosh}
\DeclareMathOperator{\sgn}{sgn}
\DeclareMathOperator{\Real}{Re}
\DeclareMathOperator{\Imag}{Im} 


\makeatletter
\newsavebox{\@brx}
\newcommand{\llangle}[1][]{\savebox{\@brx}{\(\m@th{#1\langle}\)}%
  \mathopen{\copy\@brx\kern-0.5\wd\@brx\usebox{\@brx}}}
\newcommand{\rrangle}[1][]{\savebox{\@brx}{\(\m@th{#1\rangle}\)}%
  \mathclose{\copy\@brx\kern-0.5\wd\@brx\usebox{\@brx}}}
\makeatother

\newcommand{\C}[0]{\ensuremath{\mathbb{C}}}
\newcommand{\CPone}[0]{\ensuremath{\C{}P^1}}
\newcommand{\extC}[0]{\ensuremath{\overline{\C}}}

\newcommand{\betweenSymbol}{\raisebox{0.4ex}{\ensuremath{\mathord{\includegraphics[width=0.7em]{Images/between_symbol.eps}}}}}

\newcommand{\bT}[3]{{#1} \betweenSymbol {#2} \betweenSymbol {#3}}
\newcommand{\congT}[2]{{#1} \mathbin{\equiv} {#2}}
\newcommand{\imp}{\Rightarrow}

\captionsetup[subfloat]{labelformat=empty}

\newcommand{\smallfigs}[2]{\centering
\scalebox{0.8}{\subfloat[]{#1}}
\qquad
\scalebox{0.8}{\subfloat[]{#2}}}

\usepackage{amsthm}
\theoremstyle{definition}
\newtheorem{definition}{Definition}[section]
\newtheorem{theorem}{Theorem}[section]
\newtheorem{corollary}{Corollary}[theorem]
\newtheorem{lemma}[theorem]{Lemma}
\newcommand{\norm}[1]{\left\lVert#1\right\rVert}
%\smartqed 

\begin{document}
\begin{abstract}
  In this paper we describe an Isabelle/HOL formalization of
  non-commu\-ta\-ti\-ve and non-associative algebraic structures
  called \emph{gyrogroups} and \emph{gyrovector spaces}. These were
  introduced by Abraham A. Ungar and have deep connections to
  hyperbolic geometry and special relativity. Gyrovector spaces can be
  used to define models of hyperbolic geometry. In contrast to other
  models, gyrovector spaces have the advantage that all definitions
  have remarkable syntactical similarities with normal Euclidean and
  Cartesian geometry (e.g., points on line between $a$ and $b$ satisfy
  the equation $a \oplus t\otimes(\ominus a \oplus b)$, while the
  hyperbolic Pythagorean theorem is given by $a^2\oplus b^2 = c^2$,
  where $\otimes$, $\oplus$, and $\ominus$ are gyro operations).

  We first formally define gyrogroups and gyrovector spaces, and prove
  their numerous properties. Then we formalize M\"obius and Einstein
  models of these abstract structures and then formally prove that
  these are equivalent to Poincar\'e and Klen-Beltrami models and
  satisfy Tarski's geometry axioms for hyperbolic geometry,
\end{abstract}

\section{Introduction}

Non-Euclidean geometries, including hyperbolic geometry, have been
studied since the 19th century, and their properties are well
understood. Applications of hyperbolic geometry in physics have also
been extensively explored, with findings showing that hyperbolic
geometry provides the mathematical foundation for the theory of
special relativity, much as Euclidean geometry underpins classical
Newtonian and Galilean mechanics.

Analytic definitions of hyperbolic geometry are typically presented in
the extended complex plane, relying heavily on linear algebra
techniques (such as Hermitian matrices) and hyperbolic trigonometry
\cite{schwerdtfeger}. In contrast to the Cartesian approach used in
Euclidean geometry, hyperbolic geometry does not incorporate vectors
in the same way. Euclidean geometry provides a natural framework for
vectors: vector addition is associative and commutative, forming an
Abelian group, and vectors can be scaled by real numbers, creating a
vector space. The dot (inner) product and vector norm are easily
defined, establishing the Euclidean metric,
$d(A, B) = \vert\overrightarrow{AB}\vert$. However, these concepts are
not as straightforward to define in the hyperbolic plane, resulting in
analytic definitions of hyperbolic geometry that do not employ vector
spaces. This creates an inherent asymmetry between the definitions of
Euclidean and hyperbolic geometry.

In 1988.~Ungar \cite{ungar-analytic} proposed an alternative and
developed algebraic structures known as gyrogroups and gyrovector
spaces, enabling the definition and use of ``vectors'' for the
formalization of hyperbolic geometry. Although operations with these
``vectors'' do not form traditional vector spaces (vector addition,
for example, is neither commutative nor associative) the elements,
called gyrovectors, and their associated operations still satisfy
enough algebraic properties to allow for a formalism of hyperbolic
geometry that is syntactically astonishingly similar to the classical
treatment of Euclidean geometry.

Interestingly, these structures were first discovered through an
examination of Einstein's velocity addition laws, which were then
unified with classical Galilean velocity addition. This revealed a
profound connection between the special theory of relativity and
hyperbolic geometry. Experimentally discovered Thomas
precession\cite{ungar-analytic} was found as a ,,missing link''
between Einstein's velocity addition formula and the ordinary vector
addition.

In this paper we present a formalization of gyrovector spaces and
their connections with hyperbolic geometry and special relativity,
within the interactive theorem prover Isabelle/HOL. Machine-verified
proofs are gaining popularity, and with recent advances in AI, it may
only be a matter of time before they become the norm. Such proofs are
particularly valuable in error-prone areas -- we argue that
non-associative and non-commutative algebras fall into this
category. Namely, human mathematicians, accustomed to the standard
laws of groups and vector spaces, may make assumptions in
pen-and-paper proofs that seem legitimate to those not specifically
trained to think within non-commutative and non-associative
structures. Additionally, since most gyrogroup and gyrovector space
laws are expressed as equalities, automated provers, especially those
based on term rewriting, can often find proofs that are shorter and
more elegant than those produced by humans. We will support this claim
by providing several examples.

We follow the book \cite{ungar-analytic}, but filling in many
important details that are missing in the book. This includes making
the underlying notation more precise, but also providing some proofs
missing in the book.

The paper is structured as follows. First, we describe a formalization
of gyrogroups (Section \ref{sec:gyrogroups}) and gyrovector spaces
(Section \ref{sec:gyrovectorspaces}), defining these abstract
structures and proving their important properties. In Section
\ref{sec:mobiuseinstein}, we formally demonstrate that M\"obius
transformations, traditionally used for defining hyperbolic geometry,
give rise to gyrogroup and gyrovector space structures (formalizing
the so-called M\"obius gyrovector spaces). We also formally prove that
Einstein's velocity addition satisfies the same structure (formalizing
the so-called Einstein gyrovector spaces). In Section
\ref{sec:models}, we use gyrovector spaces to define hyperbolic
geometry in a syntactically elegant manner. We then prove that these
definitions yield models equivalent to the Poincar'e and
Klein-Beltrami disks. Finally, in Section \ref{sec:conclusions}, we
draw conclusions and propose possible directions for further work

\subsection{Related work}

Over the past decade, significant progress has been made in
formalizing various geometries and their models using proof
assistants.

GeoCoq\footnote{https://geocoq.github.io/GeoCoq/}, developed by
Narboux et al., is a formalization of geometry using the Coq proof
assistant. It contains a formalization of a systematic development of
Euclidean geometry based on Tarski's axioms
\cite{tarski,narboux-tarski}. Magaud, Narboux and Schreck investigated
how projective plane geometry can be formalized in a proof assistant
\cite{coq-projective}. Narboux, Boutry and Braun used a proof
assistant Coq to formaly prove that Tarski's axioms for plane neutral
geometry can be derived from the corresponding Hilbert's
axioms\cite{hilbert-to-tarski}. They have also mechanized the proof
that Tarski's version of the parallel postulate is equivalent to the
Playfair's postulate used by Hilbert\cite{coq-parallels} and proved
that Hilbert's axioms for plane neutral geometry (excluding
continuity) can be derived from the corresponding Tarski’s
axioms\cite{tarski-to-hilbert}.

Petrovi\'c and Mari\'c describe formalization of analytic geometry of
the Cartesian plane in Isabelle/HOL \cite{adg-analytic}. Boutry, Braun
and Narboux describe the formalization of the arithmetization of
Euclidean plane geometry in the Coq proof assistant, developing
analytic geometry of the Cartesian within the Tarski's axiom
system\cite{aritmetization}.

Makarios formalized Klein/Beltrami disc model of hyperbolic geometry
in the projective space $\mathbb{R}P^2$ in
Isabelle/HOL\cite{makarios}. Based on the work of Makarios, Harrison
has shown the independence of Euclid's axiom in HOL-Light, and
Coghetto formalized the Klein-Beltrami model within
Mizar\cite{harrison2005hol,coghetto2018klein}. Mari\' c and Simi\' c
formally addressed the geometry of complex numbers and presented a
fully mechanically verified development within the theorem prover
Isabelle/HOL\cite{amai-complexplane}. Simi\' c, Mari\' c and Boutry
developed a formalization of the Poincar\'e disc model of hyperbolic
geometry within the Isabelle/HOL proof
assistant\cite{amai-poincare}. They showed that model is defined
within the complex projective line $\mathbb{C}P^1$ and adheres to
Tarski’s axioms, except for Euclid’s axiom, which it negates while
satisfying the existence of limiting parallels. The formalization
follows Schwerdfeger's book\cite{schwerdtfeger} based on linear
algebra (points are represented by vectors of complex homogenous
coordinates, circles and lines are represented by Hermitean matrices,
and M\"obius transformations by non-degenerate matrices). Our paper
builds upon this work, leveraging some of the established
results. This connection is elaborated further in Section
\ref{sec:models}.

\subsection{Background}\label{sec:background}

In this section we will describe
Isabelle/HOL\footnote{https://isabelle.in.tum.de/} and its main
features that are used for our formalization.

Isabelle is a generic proof assistant that supports various logics,
with the most popular being HOL (Higher-Order Logic). Users specify
mathematical theories, defines new concepts, states their properties
and provides proofs written in a specialized, declarative proof
language called Isar\cite{isar}. These proofs are checked by the
system, guaranteeing their correctness.

Algebraic structures are formalized using
locales\cite{isabelle-locales} and classes\cite{isabelle-classes}. In
Isabelle/HOL, \emph{locales} are particularly useful for defining and
reasoning about algebraic structures such as groups, rings, and vector
spaces, where common assumptions (like associativity, identity
elements, and distributivity) are essential across different proofs. A
locale allows users to define a set of operations and assumptions that
characterize an algebraic structure and then apply these assumptions
across multiple contexts without repeating foundational setup. Within
a locale, the \textbf{fixes} keyword is used to declare parameters,
like specific operations or elements (e.g., a binary operation
$\oplus$ and an identity element $e$ in a group). The \textbf{assumes}
keyword is used to specify the properties or axioms these parameters
must satisfy (e.g., associativity of $\oplus$ and the identity
property of $0$). The \textbf{interpretation} command is used to
instantiate a locale with specific parameters and assumptions (for
example to show that integers form a group when neutral $e$ is
interpreted by $0$ and the group operation $\oplus$ is interpreted by
integer addition $+$), effectively importing theorems and definitions
from the locale into a new context. This allows the user to apply the
abstract properties and results of an algebraic structure to concrete
examples, enabling seamless reuse of previously proven theorems under
the given interpretation.

Algebraic structures that have just a single operation are specified
using \emph{type classes} (very similar to Haskell)
\cite{isabelle-classes}. Classes are very similar to locales but the
main difference is that classes are suitable for defining type-based
properties and structures, while locales are more general and flexible
for organizing proofs and theories with shared assumptions and
parameters.

The \emph{lifting/transfer package}\cite{isabelle-lifting-transfer} is
used to transfer definitions and theorems from the raw type to the
abstract type. In Isabelle/HOL, an abstract type is a type defined in
terms of properties and behaviors rather than concrete
representation. Examples of abstract types include subset types,
quotient types, etc. Subset types are defined using \textbf{typedef}
construction. For example, the set of points in the Poincar\'e disc
can be defined as a subtype of complex numbers, containing the
elements whose norm is less than 1. Two functions are automatically
generated (assuming the concrete type is called $B$ and the abstract
type is called $A$): $\mathit{Rep\_A}$ and $\mathit{Abs\_A}$. The role
of $\mathit{Rep\_A}$ is to map an element of the new, abstract type to
its representation in the underlying type (for example,
$\mathit{Rep\_Poincare}$ maps a point of the Poincar\'e disc into its
representing complex number). $\mathit{Abs\_A}$ is essentially the
inverse function of $\mathit{Rep\_A}$ whose role is to map an element
of the underlying type to an element of the new, abstract type,
provided that the element satisfies the predicate defining the subset
(for example, $\mathit{Abs\_Poincare}$ maps a complex number with norm
less than one into a point of the Poincar\'e disc).

Functions on the abstract type can be defined by first defining
functions on the underlying representation type, and then lifting
those definitions to the abstract type, by means of lifting/transfer
package\cite{isabelle-lifting-transfer}. Such definition use the
keyword \textbf{lift\_definition}. For example, functions that operate
on points in the Poincar\'e disc can be defined by defining functions
that operate on complex numbers, and then lifting them to the subtype.
Proofs about such functions can be done by transfering them from the
abstract type to the representation type.

\section{Gyrogroups}\label{sec:gyrogroups}

In this section we give a definition of a gyrogroup, a nonassociative
group-like structure that was discovered by
Ungar\cite{ungar-analytic}. In the same way that groups serve as a
bridge between associative algebra and Euclidean geometry, gyrogroups
serve as a bridge between nonassociative algebra and hyperbolic
geometry and relativistic physics. They are the foundation on which
analytic hyperbolic geometry rests.

\begin{definition} \textbf{(Grupoid)} A structure $(G, \oplus)$ where
  $\oplus$ is a binary operation is a \emph{grupoid} if $G$ is a
  non-empty set closed for $\oplus$.
\end{definition}

\begin{definition} \textbf{(Automorphism)} An \emph{automorphism} of a
  groupoid $(G , \oplus)$ is a bijective self-map of $S$ ,
  $ \phi: G \rightarrow G$ , which preserves its groupoid operation,
  that is, $\phi (a\oplus b) = \phi (a) \oplus \phi (b)$ for all
  $a, b\in G$.
\end{definition}

\begin{definition} \textbf{(Gyrogroups \cite{ungar-analytic},
    Definition 2.5)} A grupoid $(G, \oplus)$ is a \emph{gyrogroup} if
  its binary operation satisfies the following axioms.
\begin{itemize}
\item[(G1)] There is at least one element in $G$ called a left
  identity ($0$) that satisfies $0\oplus a = a$ for all $a\in G$.
  
\item[(G2)] For each element $a\in G$ there exists an element
  $\ominus a\in G$ called a left inverse such that
  $\ominus a \oplus a = 0$.
  
\item[(G3)] For any $a, b, c \in G$ there exists a unique element
  $gyr[a,b]c$ from $G$ such that a left gyroassociative law is
  obeyed. It means that:
  
  $$a\oplus (b \oplus c) = (a\oplus b) \oplus gyr[a,b]c$$
  
\item[(G4)] The map $gyr[a, b] : G \rightarrow G$ given by
  $c \mapsto gyr[a, b]c$ is an automorphism of the groupoid
  $(G, \oplus)$, that is,

  $$gyr[a, b] \in Aut(G, \oplus)$$
  
  and the automorphism $gyr[a, b]$ of $G$ is called the
  gyroautomorphism, or the gyration, of $G$ generated by $a$,
  $b \in G$. The operator
  $gyr : G \times G \rightarrow Aut(G, \oplus)$ is called the gyrator
  of $G$.
\item[(G5)] Finally, the gyroautomorphism $gyr[a, b]$ generated by any
  $a, b \in G$ possesses the left reduction property:
  
  $$gyr[a, b] = gyr[a\oplus b, b]$$ 
\end{itemize}
\end{definition}

From these axioms, it can be easily shown that there is an unique
identity $0$ (both left and right) and that every element $a\in G$ has
its own unique inverse (both left and right). The structure of a
gyrogroup is a very rich structure, and many lemmas and theorems
follow from these axioms. We give example of one simple (but not
trivial) lemma with a proof.

\begin{lemma}\label{lemma:left_cancel} \textbf{(General left cancellation law):} Let
  $(G,\oplus)$ be a gyrogroup. For any elements $a$, $b$, $c\in G$ we
  have: If $a\oplus b=a\oplus c$ then $b=c$.
\end{lemma}
\begin{proof} From $(G2)$ we know that there must exist an element $x$
  such that $x\oplus a = 0$ where $0$ is a left identity. From $(G3)$
  we conclude that
  $x\oplus (a\oplus b) = (x \oplus a) \oplus gyr[x,a]b$ which is also
  equal to $gyr[x,a]b$ since $0$ is a left identity. On the other
  hand, $x\oplus (a\oplus b)=x\oplus (a\oplus c)$ because
  $a\oplus b=a\oplus c$ is a condition of this lemma. Moreover, from
  $(G3)$ we also know that
  $x\oplus (a\oplus c) = (x \oplus a) \oplus gyr[x,a]c =
  gyr[x,a]c$. So, finally, we have that $gyr[x,a]b=gyr[x,a]c$. From
  this equation the result follows very easy, since gyroautomorphisms
  are bijective functions.
\end{proof}

\subsection{Isabelle/HOL formalization}

Formalization of gyrogroups in Isabelle/HOL is very
straightforward. Since it is an algebraic operation on a single type
we define it as a type class\cite{isabelle-classes}. First we define a
groupoid and a groupoid automorphism.

\begin{small}
{\tt
\begin{tabbing}
\hspace{5mm}\=\kill
{\bf class} gyrogroupoid =\\
\> {\bf fixes} gyrozero :: "'a ("$0_g$")"\\
\> {\bf fixes} gyroplus :: "'a  $\Rightarrow$  'a  $\Rightarrow$  'a" (infixl "$\oplus$" 100)"\\
{\bf begin}\\[1mm]
{\bf definition} gyroaut :: "('a $\Rightarrow$ 'a) $\Rightarrow$ bool {\bf where}\\
\>    "gyroaut f $\longleftrightarrow$ 
       ($\forall$ a b. f (a $\oplus$ b) = f a $\oplus$ f b) $\land$ 
       bij f"\\[1mm]
       {\bf end}
\end{tabbing}
}
\end{small}

Next we define a gyrogroup. We define subtraction using addition and
left inverse.
     
\begin{small}
{\tt
\begin{tabbing}
\hspace{5mm}\=\kill
{\bf class}  gyrogroup  =  gyrogroupoid  +\\
\> {\bf fixes} gyroinv :: "'a $\Rightarrow$ 'a" ("$\ominus$")\\
\> {\bf fixes} gyr :: "'a $\Rightarrow$ 'a $\Rightarrow$ 'a $\Rightarrow$ 'a" \\
\> {\bf assumes} gyro\_left\_id: "\= $\bigwedge$ a. ($0_g$ $\oplus$ a = a)"\\
\> {\bf assumes} gyro\_left\_inv: "\=$\ominus$ a $\oplus$ a = $0_g$"\\
\> {\bf assumes} gyro\_left\_assoc: \\ \tab \tab "$\bigwedge$ a b z. a $\oplus$ (b $\oplus$ z) = (a $\oplus$ b) $\oplus$ (gyr a b z)"\\
\> {\bf assumes} gyr\_gyroaut: "$\bigwedge$ a b. gyroaut (gyr a b)"\\
\> {\bf assumes} gyr\_left\_loop: "$\bigwedge$ a b. gyr a b = gyr (a $\oplus$ b) b"\\
{\bf begin}\\[1mm]
{\bf definition} gyrominus :: "'a $\Rightarrow$ 'a $\Rightarrow$ 'a" (infixl "$\ominus_b$" 100) {\bf where}\\
\>    "a $\ominus_b$ b = a $\oplus$ ($\ominus$ b)"\\[1mm]
{\bf end}
\end{tabbing}
}
\end{small}


Our particullar interest is in gyrocommutative gyrogroups, since, as
we will see, some of them give a rise to gyrovector spaces.


\begin{small}
{\tt
\begin{tabbing}
\hspace{5mm}\=\kill
{\bf class} gyrocommutative\_gyrogroup = gyrogroup + \\
\> {\bf assumes}  gyro\_commute: "a $\oplus$ b = gyr a b (b $\oplus$ a)"
\end{tabbing}
}
\end{small}

Lemma \ref{lemma:left_cancel} can be formulated and proved in the Isar language as follows:

\begin{small}
{\tt
\begin{tabbing}
\hspace{5mm}\=\hspace{5mm}\=\kill
{\bf lemma} gyro\_left\_cancel:\\
\>  {\bf assumes} "a $\oplus$ b = a $\oplus$ c"\\
\>  {\bf shows} "b = c"\\
{\bf proof}-\\
\>  {\bf from} assms {\bf have} "($\ominus$a) $\oplus$ (a $\oplus$ b) = ($\ominus$a) $\oplus$ (a $\oplus$ c)" {\bf by} simp\\
\>  {\bf then} {\bf have} "($\ominus$a $\oplus$ a) $\oplus$ gyr ($\ominus$a) a b = ($\ominus$a $\oplus$ a) $\oplus$ gyr ($\ominus$a) a c"\\
\>\>    {\bf using} gyro\_left\_assoc {\bf by} simp\\
\>  {\bf then} {\bf have} "gyr ($\ominus$a) a b = gyr ($\ominus$a) a c" {\bf by} simp\\
\>  {\bf then} {\bf show} "b = c" {\bf using} gyr\_inj {\bf by} blast\\
{\bf qed}
\end{tabbing}
}
\end{small}

However, good support for automated theorem proving in Isabelle/HOL
can provide a much shorter proof than the one given in the
book. Namely, the prover metis shows the lemma fully automatically,
given a list of axioms and previous lemmas used in that automated
proof. This proof (including the relevant list of lemmas) is found by
the tool Sledgehammer\cite{sledgehammer}.

\begin{small}
{\tt
\begin{tabbing}
\hspace{5mm}\=\kill
{\bf lemma} gyro\_left\_cancel:\\
\>  {\bf assumes} "a $\oplus$ b = a $\oplus$ c"\\
\>  {\bf shows} "b = c"\\
{\bf using} assms\\
{\bf by} (metis gyr\_inj gyro\_left\_assoc gyro\_left\_id gyro\_left\_inv)
\end{tabbing}
}
\end{small}

We have formally proved a large number of lemmas and theorems about
gyrogroups (given in Chapter 2 and 3 in \cite{ungar-analytic}). Our
experience shows that the support for automated theorem proving in
Isabelle/HOL, especially in the domain of equational reasoning that
lies in the core of the proofs of gyrogroup properties, and the
Sledgehammer tool for proof finding are very powerful, so we managed
to prove many theorems that have long elaborate proofs in the book
fully automatically (of course, such proofs do not give understanding
why the statement holds).

\section{Gyrovector spaces}\label{sec:gyrovectorspaces}

Gyrogroups are used to define gyrovectors and gyrovector spaces.

\begin{definition} \textbf{(Real Inner Product Gyrovector Spaces, Definition 6.2 in
    \cite{ungar-analytic})} A real inner product gyrovector space
  $(G, \oplus, \otimes)$ (gyrovector space, in short) is a
  gyrocommutative gyrogroup $(G, \oplus)$ that obeys the following
  axioms:
\begin{itemize}
\item[(1)] $G$ is a subset of a real inner product vector space $V$
  called the \emph{carrier} of $G$, $G \subseteq V$, from which it
  inherits its inner product, $\cdot$, and norm, $\lVert \cdot \rVert$
  which are invariant under gyroautomorphisms, that is,
  
  $$gyr[u, v ] a \cdot gyr [ u , v]b = a \cdot b$$

  for all points $a, b, u, v \in G$.
  
\item[(2)] $G$ admits a \emph{scalar multiplication}, $\otimes$ ,
  possessing the following properties. For all real numbers $r$,
  $r_1$,
  $r_2 \in \mathbb{R}$ and all points $a \in G$:\\[2mm]
  \begin{tabular}{cll}
    (V1) & $1\otimes a = a$ & \\[1mm]
    (V2) & $(r1 + r2) \otimes a = (r1 \otimes a) \oplus (r2 \otimes a)$ & Scalar Distributive Law\\[1mm]
    (V3) & $(r_1r_2)\otimes a = r_1 \otimes (r_2 \otimes a)$ & Scalar Associative Law\\[1mm]
    (V4) & $\frac{|r|\otimes a}{\lVert r \otimes a \rVert} = \frac{a}{\lVert a \rVert}$ & Scaling Property\\[1mm]
    (V5) & $gyr[u,v](r\otimes a) = r \otimes gyr[u,v]a$ & Gyroautomorphism Property\\[1mm]
    (V6) & $gyr[r_1\otimes v, r_2 \otimes v] = I$ & Identity Automorphism\\
  \end{tabular}

\item[(3)] Real vector space structure
  ($\lVert G \rVert, \oplus, \otimes$) for the set $\lVert G \rVert$
  of one-dimensional "vectors":

  $$\lVert G \rVert = \{\pm \lVert a \rVert:a\in G\}\subseteq \mathbb{R}$$
  
  with vector addition $\oplus$ and scalar multiplication $\otimes$, such that for all $r\in \mathbb{R}$ and $a, b \in G$,\\[2mm]
  \begin{tabular}{cll}
    (V7) & $\lVert r\otimes a \rVert = |r| \otimes \lVert a \rVert$ & Homogeneity Property\\[1mm]
    (V8) & $\lVert a \oplus b \rVert \leq \lVert a \rVert \oplus \lVert b \rVert$ & Gyrotriangle Inequality\\ 
  \end{tabular}
\end{itemize}
\end{definition}

\subsection{Isabelle/HOL formalization}

The use of operations might look ambiguous in some axioms, and this
must be clarified in the formalization.

The first unclear place is the axiom $V4$. It uses a division -- an
operation which is at first sight not present in the structure
$(G, \otimes, \oplus)$. A closer look reveals that, since $G$ must be
a subset of some real inner product vector space $V$, ,,vectors'' from
$G$ can be multiplied by scalars from the field (in this case the
field is $\mathbb{R}$) and also by their inverses (which explains a
division in axiom $V4$). Note that the denominators
$\lVert r \otimes a \rVert$ and $\lVert a \rVert$ are real
numbers. Therefore, $V4$ uses both the gyro scalar multiplication
$\otimes$ (of $|r| \in \mathbb{R}$ and $a \in G$) and the scalar
multiplication of the inner product space (of
$\frac{1}{\lVert r \otimes a \rVert} \in \mathbb{R}$ and
$|r|\otimes a \in G$).

Operands of $\oplus$ and $\otimes$ in axioms $V7$ and $V8$ are real
numbers. In these axioms these real numbers (norms of some
gyrovectors) are treated as elements of $G$, which is in accordance to
the requirement that the set
$\Vert G \rVert = \{\pm \lVert a \rVert:a\in G\}$ is a vector space
wrt. the operations $\oplus$ and $\otimes$. Therefore, real numbers
must be embedded into $G$, i.e., there must be exist a conversion from
real numbers to elements of $G$. There is a very natural way to do
that. Since all $n$-dimensional real inner product spaces are
isomorphic to $\mathbb{R}^n$, a real number $x$ can be converted into
a vector $(x, 0, \ldots, 0)$. Note that that such embedding is
traditionally present in the expositions of Euclidean geometry (the
same notation $+$ and $\cdot$ is used for addition and scalar
multiplication of vectors and addition and scalar multiplication of
their norms -- norms are a real number and they are identified with
one-dimensional vectors). Also note that in gyrovector spaces the
operations $\oplus$ and $\otimes$ have different properties when they
operate on $G$ and on $\lVert G\rVert$ (for example, in
$\lVert G \rVert$ case addition $\oplus$ is associative and
commutative).

With this in mind, we can formalize gyrovector spaces. To formalize
relationship between $G$ and its carrier $V$ (it holds that
$G \subseteq V$) we use explicit embedding function
$\mathit{to\_carrier}$ that maps elements of $G$ to their formal
counterparts in $V$. It's inverse function (on the part of $V$ that
corresponds to $G$) is denoted by $\mathit{of\_carrier}$. The function
$\mathit{in\_domain}$ checks if the given element is in the part of
$V$ that corresponds to $G$. Norm and inner product are inherited from
the carrier space $V$.

\begin{small}
{\tt
\begin{tabbing}
  \hspace{5mm}\=\hspace{5mm}\=\kill
  {\bf locale} gyrocarrier' = \\
\>  {\bf fixes} to\_carrier :: "'a::gyrocommutative\_gyrogroup $\Rightarrow$ 'b::real\_inner"\\
\>  {\bf fixes} of\_carrier :: "'b $\Rightarrow$ 'a"\\
\>  {\bf fixes} in\_domain :: "'b $\Rightarrow$ bool"\\
\>  {\bf assumes} to\_carrier:\\
\>\>"$\bigwedge$ b. in\_domain b $\Longrightarrow$ to\_carrier (of\_carrier b) = b"\\
\>  {\bf assumes} of\_carrier: "$\bigwedge$ a. of\_carrier (to\_carrier a) = a"\\
\>  {\bf assumes} to\_carrier\_zero: "to\_carrier $0_g$ = $0$"\\
{\bf begin}\\[1mm]
{\bf definition} gyronorm :: "'a $\Rightarrow$ real" ("$\llangle$\_$\rrangle$" [100] 100) {\bf where}\\
\>    
  "$\llangle$a$\rrangle$ = norm (to\_carrier a)"\\
 {\bf definition} gyroinner :: "'a $\Rightarrow$ 'a $\Rightarrow$ real" (infixl "$\cdot$" 100) {\bf where}\\
\>    
"a $\cdot$ b = inner (to\_carrier a) (to\_carrier b)"\\
{\bf definition} norms :: "real set" {\bf where} \\
\> "norms = \{x. $\exists$ a. x = $\llangle$a$\rrangle$\} $\cup$ \{x. $\exists$ a. x = -$\llangle$a$\rrangle$\}"\\
  {\bf end}
\end{tabbing}
}
\end{small}

In a very similar manner we formalize embedding of $\lVert G\rVert$
(defined as \texttt{norms} in the previous locale) into $G$.  We
require that exist a function $\mathit{of\_real}$ that converts some
real numbers (norms of gyrovectors) to elements of $G$ (gyrovectors).
We also assume that $\mathit{to\_real}$ is its inverse function (on
its codomain).

When giving concrete interpretations of gyrovector spaces, user must
supply all embedding functions. For example, in Section
\ref{sec:mobiuseinstein} we shall see that $G$ can be the abstract set
of points in the Poincar\'e disc, $V$ is the set of complex numbers
(it is a real inner product vector space) -- the Poincar\'e disc is
identified by complex numbers that have norm less than 1.  The
function $\mathit{to\_carrier}$ maps a point in the Poincar\'e disc to
its representing complex number, and $\mathit{of\_carrier}$ is its
inverse (on the complex unit disc). The function $\mathit{of\_real}$
uses the natural embedding of real into complex numbers (it maps the
set $\lVert G\rVert$, which is in this case the open interval
$(-1, 1)$ first to complex numbers and then to points in the
Poincar\'e disc).

\begin{small}
{\tt
\begin{tabbing}
  \hspace{5mm}\=\hspace{5mm}\=\kill
{\bf locale} gyrocarrier'' = gyrocarrier' +\\
\>  {\bf fixes} of\_real :: "real $\Rightarrow$ 'a"\\
\>  {\bf fixes} to\_real :: "'a $\Rightarrow$ real"\\
\>  {\bf assumes} to\_real: "$\bigwedge$ x. x $\in$ norms $\Longrightarrow$ to\_real (of\_real x) = x"
\end{tabbing}
}
\end{small}

Next we introduce the assumption that the inner product must be
invariant under gyrations.

\begin{small}
{\tt
\begin{tabbing}
  \hspace{5mm}\=\hspace{5mm}\=\kill
{\bf class} gyrocarrier = gyrocarrier'' +  \\
\> {\bf assumes} inner\_gyroauto\_invariant: \\
\>\>"$\bigwedge$ u v a b. (gyr u v a) $\cdot$ (gyr u v b) = a $\cdot$ b"
\end{tabbing}
}
\end{small}

Finally, we define gyrovector spaces as gyrocarriers that include the
gyro scalar multiplication $\otimes$ and satisfy the given gyrovector
space axioms.

\begin{small}
{\tt
\begin{tabbing}
\hspace{5mm}\=\hspace{5mm}\=\kill
{\bf locale} gyrovector\_space = gyrocarrier +\\
\> {\bf fixes} scale ::  "real $\Rightarrow$ 'a $\Rightarrow$ 'a" (infixl "$\otimes$" 105)\\ 
\> {\bf assumes} scale\_1: "$\bigwedge$ a. 1 $\otimes$ a = a"\\
\> {\bf assumes} scale\_distrib:\\ \tab \tab "$\bigwedge$ r1 r2 a. (r1 + r2) $\otimes$ a = r1 $\otimes$ a $\oplus$ r2 $\otimes$ a"\\
\> {\bf assumes} scale\_assoc: "$\bigwedge$ r1 r2 a. (r1 * r2) $\otimes$ a = r1 $\otimes$ (r2 $\otimes$ a)"\\
\> {\bf assumes} scale\_prop1: "$\bigwedge$ r a. r $\neq$ 0 $\Longrightarrow$\\
\>\> to\_carrier (|r| $\otimes$ a) $/_R$ $\llangle$r $\otimes$ a$\rrangle$ = to\_carrier a $/_R$ $\llangle$a$\rrangle$"\\
\> {\bf assumes} gyroauto\_property: \\ \tab \tab "$\bigwedge$ u v r a. gyr u v (r $\otimes$ a) = r $\otimes$ (gyr u v a)"\\
\> {\bf assumes} gyroauto\_id: "$\bigwedge$ r1 r2 v. gyr (r1 $\otimes$ v) (r2 $\otimes$ v) = id"\\
\> {\bf assumes} homogeneity: \\ \tab \tab"$\bigwedge$ r a.  $\llangle$r $\otimes$ a$\rrangle$ =  to\_real ((abs r) $\otimes$ (of\_real $\llangle$a$\rrangle$))"\\
\> {\bf assumes} gyrotriangle: \\ \tab \tab "$\bigwedge$ a b. $\llangle$a $\oplus$ b$\rrangle$ $\leq$ to\_real (of\_real $\llangle$a$\rrangle$) $\oplus$ (of\_real $\llangle$b$\rrangle$)"\\
\end{tabbing}
}
\end{small}

In this abstract setup we proved a large number of lemmas and theorems
that hold in gyrovector spaces (given in Chapter 8 in
\cite{ungar-analytic}).

\subsection{Basic geometric notions}

The structure of gyrovector spaces is rich enough to allow defining
basic geometric notions. For $x \neq y$ it hold that $z$ belongs to
the line determined by $x$ and $y$ iff there exists a real number $t$
such that $x = y \oplus t \otimes (\ominus y \oplus z)$ (which is
fully analogous to the Euclidean case $x = y + t (z - y)$). This
allows to introduce the definition of collinearity (note that we also
include a degenerate case where $x=y$).

\begin{small}
{\tt
\begin{tabbing}
\hspace{5mm}\=\hspace{5mm}\=\kill
{\bf definition} collinear :: "'a $\Rightarrow$ 'a $\Rightarrow$ 'a $\Rightarrow$ bool" {\bf where}\\
\>"collinear x y z $\longleftrightarrow$ (y = z $\vee$ ($\exists$ t::real. (x = y $\oplus$ t $\otimes$ ($\ominus$ y $\oplus$ z))))"
\end{tabbing}
}
\end{small}

Collinearity enables the definition of lines.

\begin{small}
{\tt
\begin{tabbing}
\hspace{5mm}\=\hspace{5mm}\=\kill
{\bf definition} gyroline :: "'a $\Rightarrow$ 'a $\Rightarrow$ 'a set" {\bf where}\\
\>  "gyroline a b = \{x. collinear x a b\}"
\end{tabbing}
}
\end{small}


Definition of the betweness relation is quite similar (and again
analogous to the Euclidean case).

\begin{small}
{\tt
\begin{tabbing}
\hspace{5mm}\=\hspace{5mm}\=\kill
{\bf definition} between :: "'a $\Rightarrow$ 'a $\Rightarrow$ 'a $\Rightarrow$ bool" {\bf where}\\
\>  "between x y z $\longleftrightarrow$\\
\>\>($\exists$ t::real. 0 $\leq$ t $\wedge$ t $\leq$ 1 $\wedge$ y = x $\oplus$ t $\otimes$ ($\ominus$ x $\oplus$ z))"
\end{tabbing}
}
\end{small}

Gyro-distance between two points is defined as the norm of their
difference (again analogously to the Euclidean case).

\begin{small}
{\tt
\begin{tabbing}
\hspace{5mm}\=\hspace{5mm}\=\kill
{\bf definition} distance :: "'a $\Rightarrow$ 'a $\Rightarrow$ real" {\bf where}\\
\>  "distance u v = $\llangle \ominus$ u $\oplus$ v$\rrangle$"
\end{tabbing}
}
\end{small}

It is very easy to prove basic properties of these notions.  For
example, one very important property is that all these properties are
invariant under isometries. We define translation that maps a given
point $a$ to the origin, and prove that it preserves basic geometric
notions (proofs of these are very simple and fully automated, using
previously proven lemmas about gyrovector spaces).


\begin{small}
{\tt
\begin{tabbing}
  \hspace{5mm}\=\hspace{5mm}\=\kill
{\bf definition} translate :: "'a $\Rightarrow$ a $\Rightarrow$ a" {\bf where}\\
\>  "translate a x = $\ominus$ a $\oplus$ x"\\
{\bf lemma} collinear\_translate:\\
\>  {\bf shows} "collinear u v w $\longleftrightarrow$\\
\>\> collinear (translate a u) (translate a v) (translate a w)"\\
{\bf lemma} between\_translate:\\
\>  {\bf shows} "between u v w $\longleftrightarrow$\\
\>\> between (translate a u) (translate a v) (translate a w)"\\
{\bf lemma} distance\_translate:\\
\>  {\bf shows} "distance u v = distance (translate a u) (translate a v)"
\end{tabbing}
}
\end{small}




\section{Instances of gyrovector spaces}\label{sec:mobiuseinstein}

So far gyrogroups and gyrovector spaces have been presented as
abstract algebraic structures. In this section we shall describe their
two most important instances: M\"obius gyrovector space coming from
the realm of hyperbolic geometry and Einstein gyrovector space coming
from the realm of relativistic physics. In order to define these
spaces we must define the operations $\oplus$ and $\otimes$.


\subsection{M\"obius gyrogroup}

M\"obius gyrovector space comes from the Poincar\'e disc model of
hyperbolic geometry. In \cite{amai-poincare} it was formally proved
that all direct isometries of the Poincar\'e disc (given in the
complex plane) are of the form:

$$z \mapsto e^{i\theta} \frac{a+z}{1+\overline{a}z}.$$

This inspires the following definition of addition.

\begin{definition}\textbf{(M\" obius addition in the Poincar\'e disc)}: Let 
  $V_1 = \{z\in \mathbb{C}: \norm{z} < 1\}$ be a unit disc in
  $\mathbb{C}$. M\" obius addition $\oplus_m$ is a binary operation
  defined in $V_1$ as:
  
  $$u\oplus_m v = \frac{u+v}{1+\overline{u}v}$$
\end{definition}

Then $z \mapsto a \oplus_m z$ is a translation of the Poincar\'e disc
that maps $-a$ to zero\footnote{Such translations are sometimes called
  Blaschke factors.}, $z \mapsto e^{i\theta}z$ is a rotation of the
disc for the angle $\theta$ and each direct isometry is a translation
followed by a rotation.

Note that M\"obius addition can be naturally generalized from the
Poincar\'e disc (unit disc in $\mathbb{C}$) to a unit ball in any
finite dimension real inner product vector space $V$ (all of those
are isomorphic to $\mathbb{R}^n$). However, we did not consider this
in our formalization.

\begin{definition}\textbf{(M\" obius addition in the unit ball of $\mathbb{R}^n$)}: Let $V$
  be a real inner product vector space, let
  $V_{s} = \{v\in V: \norm{v}<s\}$ be a ball in $V$ and
  $V_{s=1} = \{v\in V: \norm{v}<1\}$ be a unit ball in $V$. M\" obius
  addition $\oplus_m$ is a binary operation defined in $V_{s=1}$ as:
  
$$u \oplus_m v = \frac{(1+2uv+\norm{v}^2)u+(1-\norm{u}^2)v}{1+2uv+\norm{u}^2\norm{v}^2}$$ 
\end{definition}

In our formalization we first defined the set of points in the
Poincar\'e disc as a subtype of complex numbers, and then we used the
lifting/transfer package to define functions operating on the
Poincar\'e disc by first defining them on the representation type
$\mathbb{C}$.

\begin{small}
{\tt
\begin{tabbing}
\hspace{5mm}\=\kill
{\bf typedef} PoincareDisc = "\{z::complex. cmod z < 1\}"\\
{\bf setup\_lifting} type\_definition\_PoincareDisc\\
\end{tabbing}
}
\end{small}

Norm and inner product are naturally lifted from $\mathbb{C}$ to the
Poincar\'e disc.

\begin{small}
{\tt
\begin{tabbing}
  \hspace{5mm}\=\kill
  {\bf lift\_definition} inner\_p ::\\
  \> "PoincareDisc $\Rightarrow$ PoincareDisc $\Rightarrow$ real" (infixl "$\cdot$" 100) {\bf is} inner\\
{\bf lift\_definition} norm\_p :: "PoincareDisc $\Rightarrow$ real"  ("$\llangle$\_$\rrangle$" [100] 101) {\bf is} norm
\end{tabbing}
}
\end{small}


We define the M\"obius addition by the following sequence of
definitions (first we define it on the representation type
$\mathbb{C}$ and then we lift it to the abstract type of points of the
Poincar\'e disc).

{\tt
\begin{small}
\begin{tabbing}
\hspace{5mm}\=\kill
{\bf definition} m\_oplus' :: "complex $\Rightarrow$ complex $\Rightarrow$ complex" {\bf where}\\
\tab "m\_oplus' u v = (u + v) / (1 + (cnj u)*v)"\\

{\bf lift\_definition} m\_oplus:: "PoincareDisc $\Rightarrow$ PoincareDisc $\Rightarrow$ PoincareDisc" \\
\tab (infixl "$\oplus_m$" 100) {\bf is} m\_oplus'
\end{tabbing}
\end{small}
}

Note that this definition must be followed by a proof that the result
of the defined function always belongs to the unit disc if its
arguments are in the unit disc.

{\tt
\begin{small}
\begin{tabbing}
\hspace{5mm}\=\kill
{\bf lemma} m\_oplus'\_in\_disc:\\
\>  {\bf assumes} "cmod u < 1" "cmod v < 1"\\
\>  {\bf shows} "cmod (m\_oplus' u v) < 1"
\end{tabbing}
\end{small}
}

Our proof of this lemma is not trivial, but is quite technical and
non-interesting. Note that such proofs are usually missing from the
literature (they are not even mentioned in \cite{ungar-analytic}), but
the formalization reveals that many such important properties must be
proved.

Recall that gyration function is necessary to fix the associativity
and commutativity. Therefore, in the case of M\"obius gyrogroup,
gyration is defined by

$$gyr_m[a, b] z = \frac{a\oplus_m b}{b\oplus_m a}z = \frac{1 + a\overline{b}}{1+\overline{a}b}z.$$

We formalize it by the next sequence of definitions.

{\tt
\begin{small}
\begin{tabbing}
\hspace{5mm}\=\kill
{\bf definition} m\_gyr' :: "complex $\Rightarrow$ complex $\Rightarrow$ complex $\Rightarrow$ complex" {\bf where}\\
\>  "m\_gyr' a b z = ((1 + a * cnj b) / (1 + cnj a * b)) * z"\\
{\bf lift\_definition} m\_gyr ::\\
\>"PoincareDisc $\Rightarrow$ PoincareDisc $\Rightarrow$ PoincareDisc $\Rightarrow$ PoincareDisc" is m\_gyr'
\end{tabbing}
\end{small}
}

For this definition to be valid, it must be proved that the result of
gyration is in the unit disc if all its areguments are in the unit
disc (which is easy, since $|1+a\overline{b}| = |1+\overline{a}b|$).

{\tt
\begin{small}
\begin{tabbing}
\hspace{5mm}\=\kill
{\bf lemma} m\_gyr'\_in\_disc:\\
\>  {\bf assumes} "cmod a < 1" "cmod b < 1" "cmod z < 1"\\
\>  {\bf shows} "cmod (m\_gyr' a b z) < 1"
\end{tabbing}
\end{small}
}

One of the central results of our formalization is that with such
gyration, $(V, \oplus)$ is a gyrocommutative gyrogroup, i.e., that
$\oplus_m$ satisfies all axioms of gyrogroup and the axiom of
gyrocommutativity (this last proof is very easy, since gyration is
defined so that the gyrcommutativity holds). Again, these proofs are
not present in the literature (at one point Ungar comments
,,$(V, \oplus_m)$ turns out to be a gyrocommutative gyrogroup, as one
can readily check by computer algebra''). They are quite technical and
indeed reduce to algebraic manipulations with complex numbers, but, in
order to formalize them we had to make all the computation steps
explicit (the are available in our formalization, but we do not print
them in the paper).


\subsection{M\"obius gyrovector space}

M\" obius gyrogroup $(V, \oplus_m)$ admits scalar multiplication
$\otimes_m$, turning it into the M\" obius gyrovector space
$(V_1, \oplus_m, \otimes_m)$.

\begin{definition}\textbf{(M\" obius Scalar Multiplication)}: Let
  $(V_1, \oplus_m)$ be a M\" obius gyrogroup. For $r \in \mathbb{R}$,
  $v \in V_1$, and $v \neq 0$, the M\" obius scalar
  multiplication\footnote{In section \ref{einstein_gvs} we shall see
    that the definition of scalar multiplication is exactly the same
    in the Einstein gyrovector space, so instead of $\otimes_m$ and
    $\otimes_e$ we denote this operation by $\otimes_{me}$}
  $r\otimes_{me} v$ is in $V_1$ and is given by the equation:
  
  $$r\otimes_{me} v = \frac{(1+\norm{v})^r - (1-\norm{v})^r}{(1+\norm{v})^r + (1-\norm{v})^r}\cdot\frac{v}{\norm{v}}.$$
  

\noindent For $v=0$, it holds that $r\otimes_{me} 0 = 0$.
\end{definition}

This formula is derived by generalizing the expressions for
$2\otimes v = v \oplus v$, $3\otimes v = v \oplus v \oplus v$ etc.

As expected, first we define our $\otimes_m$ operation on complex
numbers and then we lift this definition to the Poincar\'e disc.

{\tt
\begin{small}
\begin{tabbing}
{\bf definition}  me\_otimes'\_k  :: "real $\Rightarrow$ complex $\Rightarrow$ real" {\bf where}\\
\tab "me\_otimes'\_k r z = \=((1 + cmod z) powr r - (1 - cmod z) powr r) /\\
\>                  ((1 + cmod z) powr r + (1 - cmod z) powr r)"\\ 
{\bf definition} me\_otimes' :: "real $\Rightarrow$ complex $\Rightarrow$ complex" {\bf where}\\
\tab "me\_otimes' r z = (if z = 0 then 0 else m\_otimes'\_k r z * (z / cmod z)"\\
{\bf lift\_definition} me\_otimes:: "PoincareDisc $\Rightarrow$ PoincareDisc $\Rightarrow$ PoincareDisc" \\
\tab (infixl "$\otimes_{me}$" 105) {\bf is} me\_otimes'
\end{tabbing}
\end{small}
}

We didn't find the proof that $(V_1, \oplus_m, \otimes_{me})$ is a
gyrovector space in the literature (only some fragments were presented
in the books\cite{ungar-analytic}). So, another contribution of our
formalization is that this proof is given explicitly (and
additionally, its correctness is machine checked).

Very important characterization of $\otimes$ uses hyperbolic
trigonometric functions (instead of power function used in the
original definition).

{\tt
\begin{small}
\begin{tabbing}
\hspace{5mm}\=\kill
{\bf lemma} otimes'\_k\_tanh:\\
\>  {\bf assumes} "cmod z < 1"\\
\>  {\bf shows} "me\_otimes'\_k r z = tanh (r * artanh (cmod z))"
\end{tabbing}
\end{small}
}


Our proof would be much harder (or almost impossible) if we didn't use
the Lorentz factor, a crucial concept in the theory of relativity,
named after the Dutch physicist Hendrik Lorentz. It describes how
time, length, and relativistic mass change for an object moving
relative to an observer. The Lorentz factor $\gamma_u$ is given by the
equation:

$$\gamma_u = \frac{1}{\sqrt{1-\frac{\norm{u}^2}{c^2}}}$$

\noindent where $c$ is the speed of the light in vacuum.

A special property of Lorentz factor is that Lorentz factor is complex
if the norm of $u$ is bigger than $1$ and real if the norm is smaller
than $1$ and vice versa. However, we could not use this in
Isabelle/HOL since functions which define power ($sqrt$, $pow$, etc.)
are always real valued.

We defined the Lorentz $\gamma$-factor in the following way (assuming
without loss of generality that $c=1$):


{\tt
\begin{small}
\begin{tabbing}
\tab\=\tab\=\kill
{\bf definition} gamma\_factor :: "complex $\Rightarrow$ real" {\bf where}\\
\> "gamma\_factor u = \\
\>\> (if (norm u)$^2$ < 1 then  1 / sqrt (1 - (norm u)$^2$) else 0)"\\
{\bf lift\_definition} gammma\_factor\_p :: "PoincareDisc $\Rightarrow$ real" ("$\gamma_p$") {\bf is}\\
\> gamma\_factor
\end{tabbing}
\end{small}
\tt}


Most gyrovector space axioms were rather straightforward to prove, but
there were exceptions. The hardest axiom to prove was the gyrotriangle
inequality. We needed to prove that
$\norm{a\oplus_m b} \leq \norm{a} \oplus_m \norm{b}$ and we did that
using Lorentz factors and their properties.

First we proved a lemma that claims that $\gamma$-factor is a
monotonically increasing function on the iterval $[0, 1)$ and so its
inverse function.

{\tt
\begin{small}
\begin{tabbing}
\tab\=\tab\=\kill
{\bf lemma} gamma\_factor\_increase\_reverse:\\
\>  {\bf fixes} t1 t2 :: real\\
\>  {\bf assumes} "0 $\leq$ t1" "t1 < 1" "0 $\leq$ t2" "t2 < 1"\\
\>  {\bf assumes} "$\gamma$ t1 > $\gamma$ t2"\\
\>  {\bf shows} "t1 > t2"
\end{tabbing}
\end{small}
}

After that we have proved the following lemma that expresses the
$\gamma$-factor of the norm of the M\"obius sum of two points in terms
of their $\gamma$-factors.

{\tt
\begin{small}
\begin{tabbing}
\tab\=\tab\=\kill
{\bf lemma} gamma\_factor\_norm\_sum:\\
\>  {\bf shows} "\=$\gamma$ $\llangle$a $\oplus_m$ b$\rrangle$ =\\
\>\>$\gamma_p$ a * $\gamma_p$ b * cmod (1 + cnj (to\_complex a) * (to\_complex b))"
\end{tabbing}
\end{small}
}

From this, we could prove the triangle inequallity for the
$\gamma$-factors.

{\tt
\begin{small}
\begin{tabbing}
\tab\=\tab\=\kill
{\bf lemma} gamma\_factor\_triangle\_inequality:\\
\>  {\bf shows} "$\gamma$ $\llangle$a $\oplus_m$ b$\rrangle$ $\leq$ $\gamma_p$ ((of\_complex $\llangle a\rrangle$) $\oplus_m$ (of\_complex $\llangle$b$\rrangle$))"
\end{tabbing}
\end{small}
}

\noindent The triangle inequality holds from this by the monotonicity
of the inverse $\gamma$-factor.

Proof of those lemmas are available in our formalization. They are
non-trivial and rely on algebraic manipulations of expressions
(equalities and inequalities) over complex numbers.

\subsection{Einstein gyrovector space}
\label{einstein_gvs}

  
According to the Einstein's theory of relativity, as an object's speed
approaches the speed of light, its relativistic mass increases towards
infinity, requiring infinite energy to accelerate further. Therefore,
it is impossible for an object to move at or above the speed of
light. While Newtonian velocities are represented by vectors in
$\mathbb{R}^3$, Einstein velocities are represented by vectors within
a ball of radius $c$ where $c$ is the speed of light.

Einstein's addition law which describes how velocities combine in
special relativity, lead to rich nonassociative algebraic structures,
called gyrovector spaces. Ungar made Einstein's definitions more
general, allowing us to observe a ball of any real inner product
space. Without losing of the generality we can observe a unit ball. It
turns out that the Einstein gyrovector spaces are isomorphic to the
M\" obius gyrovector spaces and form a setting for a model of
hyperbolic geometry. Specially, we have formally proved our theorems
in the case when the ball is in $R^2$-isomorphic space, the complex
plane, but many of the proofs are correct in the general case and they
need just a slight modifications.

The general proof of connection formula between M\" obius and Einstein
addition is given. As we know, this is the first proof of this theorem
in the literature. It is not hard, but it's tehnically challenging and
it has some computational tricks.




\begin{definition}\textbf{(Einstein addition in the ball[])}: Let $V$ be a real inner product vector space and let $V_1 = \{v\in V: \norm{v}<1\}$ be a unit ball in $V$. Einstein addition $\oplus_E$ is a binary operation defined in $V$ as:
$$u \oplus_E v = \frac{1}{1+u\cdot v}\Big\{u+\frac{1}{\gamma_u}v+\frac{\gamma_u}{1+\gamma_u}(u\cdot v)u\Big\}$$ 
where $\gamma_u$ is Lorentz factor (called gamma factor) and $\cdot$ and $\norm{\cdot}$ are the inner product and norm that the ball $V_1$ inherits from its space $V$. 
\end{definition}

Einstein multiplication $\otimes_E$ is defined on the same way as M\" obius multiplication $\otimes_m$.

\subsection{Gyrovector space isomorphism}
We begin this subsection with the following definition.

\begin{definition} Two gyrovector spaces $(G_1, \oplus_1, \otimes_1)$ and $(G_2, \oplus_2, \otimes_2)$  are isomorphic if there exists a bijective map $\phi: G_1 \rightarrow G_2$ such that the following conditions are satisfied:
\begin{itemize}
    \item $\phi(u\oplus_1 v) = \phi(u)\oplus_2 \phi(v)$
    \item $\phi(r\otimes_1 v) = r\otimes_2 \phi(v)$
    \item $\frac{\phi(u)}{\parallel \phi(u) \parallel}\cdot \frac{\phi(v)}{\parallel \phi(v)\parallel} = \frac{u}{\parallel u \parallel}\cdot \frac{v}{\parallel v \parallel}$
\end{itemize}
\end{definition}
Nonformally, it means that $\phi$ preserves gyrovector space operations and keeps the inner product of the unit vectors invariant.
This definition proved to be very useful, because if we can formally prove that some triple is a gyrovector space, and we find a map (with the properties from def 4.1) to another triple we can easily prove that that triple is also a gyrovector space. 

\begin{lemma}$$\frac{1}{2}\otimes_m u = \frac{\gamma_u}{\gamma_u + 1} u$$
\end{lemma}

\begin{lemma}$$\gamma_{u\oplus_e v} = \gamma_u\gamma_v(1+u\cdot v)$$
\end{lemma}
\begin{theorem}Let $G_e=(V_c, \oplus_e, \otimes_e)$ and $G_m=(V_c, \oplus_m, \otimes_m)$ be respectively, the Einstein and the Mobius gyrovector spaces of the same ball $V_c$ of a same real inner product space $V$.
The following formulas are correct:
$$u\oplus_e v = 2\otimes_m (\frac{1}{2}\otimes_m u \oplus_m \frac{1}{2} \otimes_m v)$$
\end{theorem}
\begin{proof}From lemma we have:
$$\frac{1}{2}\otimes_m u = \frac{\gamma_u}{\gamma_u + 1} u$$
Similary for vector v. Using the definition of $\otimes_m$ for $\frac{1}{2}\otimes_m u$ and $\frac{1}{2}\otimes_m v$ we conclude that:
\begin{multline}
$$\frac{1}{2}\otimes_m u \oplus_m \frac{1}{2} \otimes_m v = \Big(1 + 2\cdot\frac{\gamma_u}{1+\gamma_u}\cdot\frac{\gamma_v}{1+\gamma_v}(u\cdot v) + \norm{\frac{\gamma_v}{\gamma_v+1}v}^2\Big) \cdot \frac{\gamma_u}{\gamma_u + 1}u + \\ + \frac{\Big(1-\norm{\frac{\gamma_u}{1+\gamma_u}}^2\cdot \frac{\gamma_v}{1+\gamma_v}v\Big)}{1+2\frac{\gamma_u}{1+\gamma_u}u \cdot \frac{\gamma_v}{1+\gamma_v}v + \norm{\frac{\gamma_u}{\gamma_u+1}u}^2 + \norm{\frac{\gamma_v}{1+\gamma_v}v}^2} $$
\end{multline}
On the other hand, $\frac{1}{2}\otimes_e (u \otimes_e v) = \frac{\gamma_{u \otimes_e v}}{\gamma_{u \otimes_e v} + 1}(u \otimes_e v)$. From the lemma and the definition of Einstein addition, we can conclude that:
$$\frac{1}{2}\otimes_e (u \otimes_e v) = \frac{\gamma_u \gamma_v (1+u\cdot v)}{\gamma_u \gamma_v (1+u\cdot v) + 1}\Big\{\frac{1}{1+u\cdot v}\Big( u +\frac{1}{\gamma_u}v + \frac{\gamma_u}{\gamma_u + 1}(u\cdot v)\cdot u \Big)\Big\}$$
We want to prove that the left side is equal to the right side. So, we will start with the left side and use a few transformations to get the right side.
First, we are going to notice that $$\norm{v}^2 = \frac{\gamma_v^2-1}{\gamma_v^2}$$ which is very easy to verify using the definition of gamma factor. Further, we can simplify our expression on the following way:
$$\norm{\frac{\gamma_v}{\gamma_v+1}v}^2 =\frac{\gamma_v^2}{(1+\gamma_v)^2}\cdot\norm{v}^2 = \frac{\gamma_v-1}{\gamma_v+1}$$
Our expression transforms into:
$$\frac{1}{2}\otimes_m u \oplus_m \frac{1}{2} \otimes_m v = \frac{\Big(1+\frac{2\gamma_u\gamma_v}{(1+\gamma_u)(1+\gamma_v)}(u\cdot v) + \frac{\gamma_v -1}{\gamma_v+1}\Big)\cdot \frac{\gamma_u}{1+\gamma_u}u + \Big(1-\frac{\gamma_u-1}{\gamma_u+1}\Big)\cdot \frac{\gamma_v}{\gamma_v+1}v}{1 + \frac{2\gamma_u\gamma_v}{(1+\gamma_u)(1+\gamma_v)}(u\cdot v)+\frac{\gamma_u-1}{1+\gamma_u}\cdot\frac{\gamma_v-1}{\gamma_v+1}}$$

We simplify our expression more using the following equations:
$$1+\frac{\gamma_v-1}{\gamma_v+1}=\frac{2\gamma_v}{\gamma_v+1}$$
$$1-\frac{\gamma_u-1}{\gamma_u+1}=\frac{2}{\gamma_u+1}$$

so we get:
$$\frac{1}{2}\otimes_m u \oplus_m \frac{1}{2} \otimes_m v =\frac{\Big( \frac{2\gamma_v}{1+\gamma_v}+\frac{2\gamma_u\gamma_v}{(1+\gamma_u)(1+\gamma_v)}(u\cdot v)\Big)\cdot\frac{\gamma_u}{1+\gamma_u}u + \frac{2}{1+\gamma_u}\cdot\frac{\gamma_v}{1+\gamma_v}v}{1+\frac{2\gamma_u\gamma_v}{(1+\gamma_u)(1+\gamma_v)}(u\cdot v)+\frac{\gamma_u-1}{1+\gamma_u}\cdot \frac{\gamma_v-1}{\gamma_v+1}}$$

Also, we have that:
$$1+\frac{\gamma_u-1}{\gamma_u+1}\cdot \frac{\gamma_v-1}{\gamma_v+1}= \frac{2(1+\gamma_u\gamma_v)}{(1+\gamma_u)(1+\gamma_v)}$$

Now, we can reduce to a common denominator and get:
$$1+\frac{2\gamma_u\gamma_v}{(1+\gamma_u)(1+\gamma_v)}(u\cdot v)+\frac{\gamma_u-1}{1+\gamma_u}\cdot \frac{\gamma_v-1}{\gamma_v+1} = \frac{2\gamma_u\gamma_v(1+u\cdot v) + 2}{(1+\gamma_u)(1+\gamma_v)}$$ 

On the other side, the numerator of the fraction representing $\frac{1}{2}\otimes_m u \oplus_m \frac{1}{2} \otimes_m v$ is equal to:
$$2\gamma_u\gamma_v\Big(u+\frac{1}{\gamma_u}v+\frac{\gamma_u}{1+\gamma_u}(u \cdot v)u\Big)$$

Putting this together, we get our formula.

\end{proof}

This wasn't enough to prove that structure $(V_1, \oplus_e, \otimes_e)$ is a gyrovector space. We needed to prove how their gyrations are connected so we formulated and proved the following lemmas:

{\tt
\begin{footnotesize}
\begin{tabbing}
{\bf lemma} e\_gyr\_m\_gyr:\\
{\bf shows} \="(1/2) $\otimes_E$ e\_gyr u v w = m\_gyr ((1/2) $\otimes_E$ u) ((1/2) $\otimes_E$ v) ((1/2) $\otimes_E$ w) "\\
{\bf lemma} m\_gyr\_e\_gyr:\\
{\bf shows} \="2 $\otimes_E$ m\_gyr u v w = e\_gyr (2 $\otimes_E$ u) (2 $\otimes_E$ v) (2 $\otimes_E$ w)"\\
\end{tabbing}
\end{footnotesize}
}

The proof of the second followed immediately from the first one.

We made an effort to formally prove the connection between M\" obius and Einstein structures in complex plane. The benefits are numerous; we were able to prove that Einstein structure $(V_1,\oplus_E, \otimes_E)$ is a gyrovector space without unfolding the definitions of $\oplus_E$ and $\otimes_E$ and dealing with lots of hard calculations.


Here is the example of one of the lemma who has a short proof thanks to the connection formula we have proved before:

{\tt
\begin{footnotesize}
\begin{tabbing}
{\bf lemma} e\_gyro\_left\_assoc:\\
{\bf shows} \="a $\oplus_E$ (b $\oplus_E$ z) = (a $\oplus_E$ b) $\oplus_E$ e\_gyr a b z "\\
{\bf proof-}\\
\tab {\bf have} "(1/2) $\otimes_E$ (a $\oplus_E$ (b $\oplus_E$ z)) = ((1/2) $\otimes_E$ a) $\oplus_m$((1/2) $\otimes_E$ (b $\oplus_E$ z) )" 
\\ \tab \tab by $\ldots$\\

\tab {\bf moreover have} " ((1/2) $\otimes_E$ a) $\oplus_m$((1/2) $\otimes_E$ (b $\oplus_E$ z)) = \\ \tab \tab ((1/2) $\otimes_E$ a) $\oplus_m$((1/2) $\otimes_E$ b $\oplus_m$ (1/2) $\otimes_E$ z)" 
\\ \tab \tab by $\ldots$\\

\tab {\bf moreover have} "((1/2) $\otimes_E$ a) $\oplus_m$((1/2) $\otimes_E$ b $\oplus_m$ (1/2) $\otimes_E$ z) = \\ \tab \tab ((1/2) $\otimes_E$ a $\oplus_m$ (1/2) $\otimes_E$ b) $\oplus_m$ \\ \tab \tab m\_gyr ((1/2) $\otimes_E$ a) ((1/2) $\otimes_E$ b) ((1/2) $\otimes_E$ z)" 
\\ \tab \tab by $\ldots$\\


\tab {\bf moreover have} "(1/2) $\otimes_E$ ((a $\oplus_E$ b) $\oplus_E$ e\_gyr a b z) \\ \tab \tab = (1/2)  $\otimes_E$ (a $\oplus_E$ b) $\oplus_m$ (1/2)  $\otimes_E$ e\_gyr a b z " 
\\ \tab \tab by $\ldots$\\

\tab {\bf moreover have} "(1/2)  $\otimes_E$ (a $\oplus_E$ b) $\oplus_m$ (1/2)  $\otimes_E$ e\_gyr a b z \\ \tab \tab= ((1/2)  $\otimes_E$ a  $\oplus_m$ (1/2)  $\otimes_E$ b)  $\oplus_m$  \\ \tab \tab m\_gyr ((1/2) $\otimes_E$ a) ((1/2) $\otimes_E$ b) ((1/2) $\otimes_E$ z)" 
\\ \tab \tab by $\ldots$\\

\tab {\bf moreover have} "(1/2) $\otimes_E$ (a $\oplus_E$ (b $\oplus_E$ z)) = \\ \tab \tab (1/2)  $\otimes_E$ ((a $\oplus_E$ b) $\oplus_E$ e\_gyr a b z)" 
\\ \tab \tab by $\ldots$\\

\tab {\bf ultimately show} "?thesis"\\
\tab \tab by $\ldots$\\
{\bf qed}\\
\end{tabbing}
\end{footnotesize}
}









- definisati ova dva konkretna prostora i ukratko opisati kako je
formalno dokazano da oni zadovoljavaju aksiome (ako je nešto
originalno urađeno tj. nema kond Ungara to posebno naglasiti)

- diskutovati onu vezu sa jednom polovinom i dva i diskutovati kako je to olakšalo dokaz (nadam se da jeste)

\section{Modelling hyperbolic geometry}\label{sec:models}

In this section we are going to explain how to show that our new gyrodefinitions are formaly equivalent to our old definitions used in paper [].

One of the models used to represent hyperbolic geometry is the Poincar\'{e} disk model. We named our type PoincareDisc on purpose, to highlight the equivalence between definitions from this paper and [].

Basic objects in the Poincare model of the hyperbolic plane are h-points and h-lines. H-points are points of the extended complex plane that lie within the unit disc. From the following definition:
\begin{definition}\textbf{(Point in M\" obius gyrovector space)} Any element of M\" obius gyrovector space is a point in that space.
\end{definition}

We can conclude that the definition of a point in M\" obius gyrovector space coincides with the definition of a h-point.

By the characterization of h-lines, we know that h-lines are Euclidean circles if they don't contain the center of a unit disk (complex $0$). Otherwise they are Euclidean lines.

We want to prove that gyrocollinearity in M\" obius gyrovector space implies h-collinearity and vice versa. Also, we are interested in proving that the order of points is unchanged (h-betweenness is equivalent to gyro-betweeness).

First, lets give some definitions:

\begin{definition}\textbf{(Gyrocollinearity in M\" obius gyrovector space)}: Three points $u$, $v$ and $x$ in M\" obius gyrovector space are gyrocollinear if there exists a real number $t$ such that:
$$x = u\oplus_m t \otimes_m (\ominus_m u \oplus_m v)$$
\end{definition}

\begin{definition}\textbf{(Betweenness in M\" obius gyrovector space):} Point $x$ is between two points $u$ and $v$ if $x$, $u$ and $v$ are gyrocollinear with coefficient $t$ between 0 and 1.
\end{definition}

In our formalization, we didn't start from these definitions, but we have proved that they are correct.

Consider three gyrocollinear points $u$, $v$ and $x$. There exists some real number $t$ such that $$x=u\oplus_m t\otimes_m (\ominus_m u \oplus_m v)$$
Without losing of generality we can assume that $u=0$. If $u$ is not $0$ we can use the following M\" obius transformation:
$$f_u(z) = \frac{z-u}{1-\overline{u}z}=\ominus_m u \oplus_m z$$ to provide that $u=0$. The function $f_u(z)$ is obviously closed on ball $V_1$.
Another very important property of $f_u(z)$ is that it keeps a gyrocollinearity.

\begin{theorem}
Three points $x$, $u$ and $v$ are gyrocollinear iff $f_a(x)$, $f_a(u)$ and $f_a(v)$ are gyrocollinear.
\end{theorem}
\begin{proof} Since three points $x$, $u$ and $v$ are gyrocollinear, there exists a real number $t$ such that $x=u\oplus_m t\otimes_m (\ominus_m u \oplus_m v)$. We can prove that this formula is equivalent to $$(\ominus_m a \oplus_m x) = (\ominus_m a \oplus_m u) \oplus_m t \otimes_m (\ominus_m (\ominus_m a \oplus_m u) \oplus_m (\ominus_m a \oplus_m v))$$
using the following equalities:
\begin{align*} 
(\ominus_m a \oplus_m u) &=  \ominus_m a \oplus_m (u\oplus_m t \otimes_m (\ominus_m u \oplus_m v)) \\ 
 &=  (\ominus_m a \oplus_m u) \oplus_m m\_gyr (\ominus_m a, u, t\otimes_m (\ominus_m u \oplus_m v))\\
 &= (\ominus_m a \oplus_m u) \oplus_m t \otimes_m m\_gyr(\ominus_m a, u, \ominus_m u \oplus_m v)\\
 &= (\ominus_m a \oplus_m u) \oplus_m t \otimes_m (\ominus_m (\ominus_m a \oplus_m u) \oplus_m (\ominus_m a \oplus_m v))
\end{align*}
\end{proof}

So, we can conclude that transforming our unit disk in a way that $u$ becomes zero, doesn't change gyrocollinearity of our points and we can continue our work assuming that $u$ is zero.

If $u$ is zero, then $x=t\otimes_m v$ which is equal to tanh(t$\cdot$ artanh($\norm{v}$))$\frac{v}{\norm{v}}$ like we have formally proved in our work. 

The argument of a complex number $x$ is equal to the argument of a complex number $z$. This follows from the fact that the argument of a purely real number is zero, and $\frac{tanh(t\cdot artanh(\norm{v}))}{\norm{v}}$ is truly a purely real number. The complex argument of a product of purely real number and complex number is the sum of their arguments. So, $arg(a\cdot z)$, where $a$ is a purely real number and $z$ is a complex number, is $arg(a)+arg(z) = arg(z)$.  

Further, the fact that $x$ and $v$ have same argument means that they are also h-collinear. Also, $u$ is h-collinear with them, because we get $u$ putting $t=0$.

If $x$ is between $u=0$ and $v$, then $0\leq t \leq 1$, so we have:
$$\frac{\norm{x}}{\norm{v}} = \frac{tanh(t\cdot artanh(\norm{v})}{\norm{v}}\leq \frac{tanh(artanh(\norm{v}))}{\norm{v}}=1$$

This inequality is true because $tanh$ is strictly increasing for all real numbers. It follows that $\norm{x} \leq \norm{v}$ and $x$ is h-between $u=0$ and $v$.


- Definisati osnovne pojmove Poincare-ovg diska u Mebijusovom žirovektorskom prostoru (tačke, prave, između, rastojanje, podudarno, Mebijusove transformacije)

- Diskutovati kako je dokazano da su osnovni pojmovi invarijantni u odnosu na Mebijustove transformacije

- Diskutovati kako je to pomoglo da se uspostavi formalna ekivalentnost između ovih žiro-definicija i starih Poincare definicija

\section{Conclusions and further work}\label{sec:conclusions}

\bibliographystyle{plain}
\bibliography{gyro} % name your BibTeX data base

\end{document}
